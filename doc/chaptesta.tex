\chapter{Test}\label{chaptest}
\markboth{\uppercase{Fichiers de test}}{\uppercase{Fichiers de test}}

\minitoc

\section{Introduction}

In bin/tests there is the file options.db3. You should copy it in the
directory bin as ''cp options.db3.. /.', and then you launch the code
after the load is appear four test configurations that allow you to
see all the options in action.


\section{Test1}

The aim of test1 is to test a simple case and many options of the code
to validate them. Figure~\ref{test1conf} shows the options of the
chosen configuration.


%%%%%%%%%%%%%%%%%%%%%%%%%%%%%%%%%%%%%%%%%%%%%%%
\begin{figure}[H]
\begin{center}
  \includegraphics*[width=15.0cm,draft=false]{test1conf.eps}
\end{center}
\caption{Test1: configuration taken.}
\label{test1conf}
\end{figure}
%%%%%%%%%%%%%%%%%%%%%%%%%%%%%%%%%%%%%%%%%%%%%%%

The following figures show the results obtained. The plots are done
with Matlab and these are directly the eps files from the ifdda.m
script that are used. The advantage of matlab in this case is to give
all the figures in one go.
%%%%%%%%%%%%%%%%%%%%%%%%%%%%%%%%%%%%%%%%%%%%%%%
\begin{figure}[H]
\begin{center}
  \includegraphics*[width=15.0cm,draft=false]{test1local.eps}
\end{center}
\caption{Modulus of the local field in $(x,y)$ plane.}
%\label{test1res1}
\end{figure}
%%%%%%%%%%%%%%%%%%%%%%%%%%%%%%%%%%%%%%%%%%%%%%%
%%%%%%%%%%%%%%%%%%%%%%%%%%%%%%%%%%%%%%%%%%%%%%%
\begin{figure}[H]
\begin{center}
  \includegraphics*[width=15.0cm,draft=false]{test1macro.eps}
\end{center}
\caption{Modulus of the macroscopic field in $(x,y)$ plane.}
%\label{test1res1}
\end{figure}
%%%%%%%%%%%%%%%%%%%%%%%%%%%%%%%%%%%%%%%%%%%%%%%
Because the incident field is polarized along the $y$ direction (TE),
hence the $y$ component of the field inside the sphere is the largest.

%%%%%%%%%%%%%%%%%%%%%%%%%%%%%%%%%%%%%%%%%%%%%%%
\begin{figure}[H]
\begin{center}
  \includegraphics*[width=15.0cm,draft=false]{test1poynting2d.eps}
\end{center}
\caption{Modulus of the Poynting vector.}
%\label{test1res1}
\end{figure}
%%%%%%%%%%%%%%%%%%%%%%%%%%%%%%%%%%%%%%%%%%%%%%%


%%%%%%%%%%%%%%%%%%%%%%%%%%%%%%%%%%%%%%%%%%%%%%%
\begin{figure}[H]
\begin{center}
\begin{tabular}{cc}
  \includegraphics*[width=7.0cm,draft=false]{test1fourierpos.eps}
&  \includegraphics*[width=7.0cm,draft=false]{test1fourierneg.eps}
\end{tabular}

\end{center}
\caption{Modulus of the diffracted field in the Fourier plane in
  transmission (left) and in reflection (right) for optical diffraction
  microscope.}
%\label{test1res2}
\end{figure}
%%%%%%%%%%%%%%%%%%%%%%%%%%%%%%%%%%%%%%%%%%%%%%%

%%%%%%%%%%%%%%%%%%%%%%%%%%%%%%%%%%%%%%%%%%%%%%%
\begin{figure}[H]
\begin{center}
\begin{tabular}{cc}
  \includegraphics*[width=7.0cm,draft=false]{test1imagepos.eps}
&  \includegraphics*[width=7.0cm,draft=false]{test1imageneg.eps} \\
\includegraphics*[width=7.0cm,draft=false]{test1imageincpos.eps}
&  \includegraphics*[width=7.0cm,draft=false]{test1imageincneg.eps}

\end{tabular}

\end{center}
\caption{Modulus of the field in the image plane in transmission
  (left) and in reflection (right) for optical diffraction tomography.
  Diffracted field (above) and total field (below).}
\end{figure}
%%%%%%%%%%%%%%%%%%%%%%%%%%%%%%%%%%%%%%%%%%%%%%%

\section{Test2}

The aim of the test2 is to test a more complex case.  Figure~
\ref{test2conf} shows the options of the chosen configuration. The
illumination is done with a Gaussian beam with $w_0=\lambda$ and the
object under study is a sphere with a radius of 500~nm and an
inhomogeneous permittivity ($l_c=100$~nm and $\sigma=0.1$).

%%%%%%%%%%%%%%%%%%%%%%%%%%%%%%%%%%%%%%%%%%%%%%%
\begin{figure}[H]
\begin{center}
  \includegraphics*[width=15.0cm,draft=false]{test2conf.eps}
\end{center}
\caption{Test2: configuration taken.}
\label{test2conf}
\end{figure}
%%%%%%%%%%%%%%%%%%%%%%%%%%%%%%%%%%%%%%%%%%%%%%%
%%%%%%%%%%%%%%%%%%%%%%%%%%%%%%%%%%%%%%%%%%%%%%%
\begin{figure}[H]
\begin{center}
  \includegraphics*[width=15.0cm,draft=false]{test2epsilon.eps}
\end{center}
\caption{Test2: relative permittivity. Real part (left) and imaginary
  part (right).}
%\label{test2conf}
\end{figure}
%%%%%%%%%%%%%%%%%%%%%%%%%%%%%%%%%%%%%%%%%%%%%%%
The following figures show the results obtained.
%%%%%%%%%%%%%%%%%%%%%%%%%%%%%%%%%%%%%%%%%%%%%%%
\begin{figure}[H]
\begin{center}
  \includegraphics*[width=15.0cm,draft=false]{test2local.eps}
\end{center}
\caption{Modulus of the local field in the $(x,y)$ plane.}
%\label{test1res1}
\end{figure}
%%%%%%%%%%%%%%%%%%%%%%%%%%%%%%%%%%%%%%%%%%%%%%%
%%%%%%%%%%%%%%%%%%%%%%%%%%%%%%%%%%%%%%%%%%%%%%%
\begin{figure}[H]
\begin{center}
  \includegraphics*[width=15.0cm,draft=false]{test2macro.eps}
\end{center}
\caption{Modulus of the macroscopic field in the $(x,y)$ plane.}
%\label{test1res1}
\end{figure}
%%%%%%%%%%%%%%%%%%%%%%%%%%%%%%%%%%%%%%%%%%%%%%%


%%%%%%%%%%%%%%%%%%%%%%%%%%%%%%%%%%%%%%%%%%%%%%%
\begin{figure}[H]
\begin{center}
  \includegraphics*[width=15.0cm,draft=false]{test2poynting2d.eps}
\end{center}
\caption{Modulus of the Poynting vector.}
%\label{test1res1}
\end{figure}
%%%%%%%%%%%%%%%%%%%%%%%%%%%%%%%%%%%%%%%%%%%%%%%


%%%%%%%%%%%%%%%%%%%%%%%%%%%%%%%%%%%%%%%%%%%%%%%
\begin{figure}[H]
\begin{center}
\begin{tabular}{cc}
  \includegraphics*[width=7.0cm,draft=false]{test2fourierpos.eps}
&  \includegraphics*[width=7.0cm,draft=false]{test2fourierneg.eps} \\
\includegraphics*[width=7.0cm,draft=false]{test2fourierincpos.eps}
&  \includegraphics*[width=7.0cm,draft=false]{test2fourierincneg.eps}
\end{tabular}

\end{center}
\caption{Modulus of the field in the Fourier plane in transmission
  (left) and in reflection (right) for an optical diffraction
  tomography microscope. Diffracted field (above) and total field
  (below).}
%\label{test1res2}
\end{figure}
%%%%%%%%%%%%%%%%%%%%%%%%%%%%%%%%%%%%%%%%%%%%%%%

%%%%%%%%%%%%%%%%%%%%%%%%%%%%%%%%%%%%%%%%%%%%%%%
\begin{figure}[H]
\begin{center}
\begin{tabular}{cc}
  \includegraphics*[width=7.0cm,draft=false]{test2imagepos.eps}
&  \includegraphics*[width=7.0cm,draft=false]{test2imageneg.eps} \\
\includegraphics*[width=7.0cm,draft=false]{test2imageincpos.eps}
&  \includegraphics*[width=7.0cm,draft=false]{test2imageincneg.eps}

\end{tabular}

\end{center}
\caption{Modulus of the field in the image plane in transmission
  (left) and in reflection (right) for an optical diffraction
  tomography microscope. Diffracted field (above) and total field
  (below).}
\end{figure}
%%%%%%%%%%%%%%%%%%%%%%%%%%%%%%%%%%%%%%%%%%%%%%%

\section{Test3}

The aim of the test3 is to test the brightfield microscope with a
sphere of radius of 500~nm and a permittivity of 1.1 put upon a glass
substrate ($\varepsilon=2.25$).

%%%%%%%%%%%%%%%%%%%%%%%%%%%%%%%%%%%%%%%%%%%%%%%
\begin{figure}[H]
\begin{center}
  \includegraphics*[width=15.0cm,draft=false]{test3conf.eps}
\end{center}
\caption{Test3: configuration taken.}
\label{test3conf}
\end{figure}
%%%%%%%%%%%%%%%%%%%%%%%%%%%%%%%%%%%%%%%%%%%%%%%

%%%%%%%%%%%%%%%%%%%%%%%%%%%%%%%%%%%%%%%%%%%%%%%
\begin{figure}[H]
\begin{center}
  \includegraphics*[width=8.0cm,draft=false]{test3angleincbf.eps}
\end{center}
\caption{Test3: Incident field used to simulate the microscope.}
%\label{test3conf}
\end{figure}
%%%%%%%%%%%%%%%%%%%%%%%%%%%%%%%%%%%%%%%%%%%%%%%

%%%%%%%%%%%%%%%%%%%%%%%%%%%%%%%%%%%%%%%%%%%%%%%
\begin{figure}[H]
\begin{center}
\begin{tabular}{cc}
  \includegraphics*[width=7.0cm,draft=false]{test3imageposwf.eps}
&  \includegraphics*[width=7.0cm,draft=false]{test3imagenegwf.eps} \\
\includegraphics*[width=7.0cm,draft=false]{test3imageincposwf.eps}
&  \includegraphics*[width=7.0cm,draft=false]{test3imageincnegwf.eps}
\end{tabular}
\end{center}
\caption{Modulus of the field in the image plane in case of
  transmission (left) and reflection (right) for a darkfield (above)
  and brightfield (below) microscope.}
\end{figure}
%%%%%%%%%%%%%%%%%%%%%%%%%%%%%%%%%%%%%%%%%%%%%%%
\section{Test4}

The aim of test4 is to test the dark field and phase microscope with a
sphere of radius of 500~nm and a permittivity of 1.1 put upon a glass
substrate ($\varepsilon=2.25$).

%%%%%%%%%%%%%%%%%%%%%%%%%%%%%%%%%%%%%%%%%%%%%%%
\begin{figure}[H]
\begin{center}
  \includegraphics*[width=15.0cm,draft=false]{test4conf.eps}
\end{center}
\caption{Test3: configuration taken.}
%\label{test3conf}
\end{figure}
%%%%%%%%%%%%%%%%%%%%%%%%%%%%%%%%%%%%%%%%%%%%%%%
%%%%%%%%%%%%%%%%%%%%%%%%%%%%%%%%%%%%%%%%%%%%%%%
\begin{figure}[H]
\begin{center}
  \includegraphics*[width=8.0cm,draft=false]{test4angleincdf.eps}
\end{center}
\caption{Test4: Incident field used to simulate the microscope.}
%\label{test3conf}
\end{figure}
%%%%%%%%%%%%%%%%%%%%%%%%%%%%%%%%%%%%%%%%%%%%%%%

%%%%%%%%%%%%%%%%%%%%%%%%%%%%%%%%%%%%%%%%%%%%%%%
\begin{figure}[H]
\begin{center}
\begin{tabular}{cc}
  \includegraphics*[width=7.0cm,draft=false]{test4imageposwf.eps}
&  \includegraphics*[width=7.0cm,draft=false]{test4imagenegwf.eps} \\
\includegraphics*[width=7.0cm,draft=false]{test4imageincposwf.eps}
&  \includegraphics*[width=7.0cm,draft=false]{test4imageincnegwf.eps}

\end{tabular}

\end{center}
\caption{Modulus of the field in the image plane in case of
  transmission (left) and reflection (right) for a darkfield (above)
  and phase (below) microscope.}
\end{figure}
%%%%%%%%%%%%%%%%%%%%%%%%%%%%%%%%%%%%%%%%%%%%%%%
