\chapter{Fichiers de test}\label{chaptest}
\markboth{\uppercase{Fichiers de test}}{\uppercase{Fichiers de test}}

\minitoc

\section{Introduction}

Dans bin/tests est dispos� un fichier options.db3. Si on le copie un
directory en dessous ``cp options.db3 ../.'', quand on lance le code
apr�s un load il appara�t quatre configurations test qui permettent de
voir toutes les options en action.

\section{Test1}

Le but du test1 est de tester un cas simple mais avec une pr�cision
importante pour l'inversion du syst�me d'�quations lin�aires 
(tol�rance=$10^{-9}$) et de nombreuses options du code afin de les valider.
La Fig.~\ref{test1conf} montre les options de la configuration choisie.


%%%%%%%%%%%%%%%%%%%%%%%%%%%%%%%%%%%%%%%%%%%%%%%
\begin{figure}[H]
\begin{center}
  \includegraphics*[width=15.0cm,draft=false]{test1conf.eps}
\end{center}
\caption{Test1: configuration choisie.}
\label{test1conf}
\end{figure}
%%%%%%%%%%%%%%%%%%%%%%%%%%%%%%%%%%%%%%%%%%%%%%%

Les Figs.~\ref{test1res1}-\ref{test1res3} montrent les r�sultats
obtenus.  Les trac�s sont effectu�s avec Matlab, mais peuvent bien s�r
�tre r�alis�s avec l'interface graphique int�gr�e.
%%%%%%%%%%%%%%%%%%%%%%%%%%%%%%%%%%%%%%%%%%%%%%%
\begin{figure}[H]
\begin{center}
  \includegraphics*[width=15.0cm,draft=false]{test1champlocalmacro.eps}
\end{center}
\caption{Module du champ local et macroscopique dans le plan $(x,y)$
  pour un $z$ donn�e passant au centre de la sph�re.}
\label{test1res1}
\end{figure}
%%%%%%%%%%%%%%%%%%%%%%%%%%%%%%%%%%%%%%%%%%%%%%%
Le champ incident �tant polaris� suivant la composante $y$ (TE), il
est clair que la composante $y$ du champ � l'int�rieur de la sph�re
est la plus forte.
%%%%%%%%%%%%%%%%%%%%%%%%%%%%%%%%%%%%%%%%%%%%%%%
\begin{figure}[H]
\begin{center}
  \includegraphics*[width=15.0cm,draft=false]{test1champtransmis.eps}
\end{center}
\caption{Intensit� en transmission pour un microscope holographique:
  champ de Fourier, champ image sans et avec le champ incident.}
\label{test1res2}
\end{figure}
%%%%%%%%%%%%%%%%%%%%%%%%%%%%%%%%%%%%%%%%%%%%%%%

%%%%%%%%%%%%%%%%%%%%%%%%%%%%%%%%%%%%%%%%%%%%%%%
\begin{figure}[h]
\begin{center}
  \includegraphics*[width=15.0cm,draft=false]{test1champreflechis.eps}
\end{center}
\caption{Intensit� en r�flexion pour un microscope holographique:
  champ de Fourier, champ image sans et avec le champ incident.}
\label{test1res3}
\end{figure}
%%%%%%%%%%%%%%%%%%%%%%%%%%%%%%%%%%%%%%%%%%%%%%%

\section{Test2}

Le but du test2 est de tester un cas plus compliqu� avec une sph�re 
pr�sentant une distribution inhomog�ne de permittivit�.  La Fig.~
\ref{test2conf} montre les options de la configuration choisie.


%%%%%%%%%%%%%%%%%%%%%%%%%%%%%%%%%%%%%%%%%%%%%%%
\begin{figure}[H]
\begin{center}
  \includegraphics*[width=15.0cm,draft=false]{test2conf.eps}
\end{center}
\caption{Test2: configuration choisie.}
\label{test2conf}
\end{figure}
%%%%%%%%%%%%%%%%%%%%%%%%%%%%%%%%%%%%%%%%%%%%%%%

La permittivit� relative de la sph�re est donc inhomog�ne, et 
l'�clairement est r�alis� par un faisceau Gaussien polaris� suivant $x$
et $z$.
%%%%%%%%%%%%%%%%%%%%%%%%%%%%%%%%%%%%%%%%%%%%%%%
\begin{figure}[H]
\begin{center}
  \includegraphics*[width=15.0cm,draft=false]{test2incident.eps}
\end{center}
\caption{Relative permittivit� et champ incident pour le test 2.}
\label{test2inc}
\end{figure}
%%%%%%%%%%%%%%%%%%%%%%%%%%%%%%%%%%%%%%%%%%%%%%%

Les Figs.~\ref{test2res1}-\ref{test2res3} montrent les r�sultats
obtenus.  
%%%%%%%%%%%%%%%%%%%%%%%%%%%%%%%%%%%%%%%%%%%%%%%
\begin{figure}[H]
\begin{center}
  \includegraphics*[width=15.0cm,draft=false]{test2champlocalmacro.eps}
\end{center}
\caption{Module du champ local et macroscopique dans le plan $(x,y)$
  pour un $z$ donn�e passant au centre de la sph�re.}
\label{test2res1}
\end{figure}
%%%%%%%%%%%%%%%%%%%%%%%%%%%%%%%%%%%%%%%%%%%%%%%
Le champ incident est un faisceau Gaussien inclin� � 30 degr�s et
polaris� suivant la direction  $x$ (il y a donc une composant $z$).
%%%%%%%%%%%%%%%%%%%%%%%%%%%%%%%%%%%%%%%%%%%%%%%
\begin{figure}[H]
\begin{center}
  \includegraphics*[width=15.0cm,draft=false]{test2champtransmis.eps}
\end{center}
\caption{Module du champ en transmission pour un microscope
  holographique: champ de Fourier, champ image sans et avec le champ
  incident.}
\label{test2res2}
\end{figure}
%%%%%%%%%%%%%%%%%%%%%%%%%%%%%%%%%%%%%%%%%%%%%%%

%%%%%%%%%%%%%%%%%%%%%%%%%%%%%%%%%%%%%%%%%%%%%%%
\begin{figure}[h]
\begin{center}
  \includegraphics*[width=15.0cm,draft=false]{test2champreflechis.eps}
\end{center}
\caption{Module en r�flexion pour un microscope holographique: champ
  de Fourier, champ image sans et avec le champ incident.}
\label{test2res3}
\end{figure}
%%%%%%%%%%%%%%%%%%%%%%%%%%%%%%%%%%%%%%%%%%%%%%%

\section{Test3}

\section{Test4}

